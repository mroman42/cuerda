\documentclass{article}

\usepackage{chorda}
\usepackage{mydiagrams}

\title{Corda \\ \normalsize{\textsf{version 0.1}}}
\author{Mario Román}

\begin{document}
\maketitle

\textbf{Corda} is a Haskell library that produces string and surface diagrams from descriptions of morphisms in monoidal categories and monoidal bicategories.
\begin{figure}[h]
  \centering
  \assocDiagram
  \caption{A diagram generated by Corda.}
\end{figure}

Corda lets you write Haskell code that outputs a TikZ library with your diagrams. This library can be called from your main LaTeX file.

\subsection*{Referencing Corda}

As of June 2020, the best way of referencing \textsf{Corda} is by pointing to its GitHub repository.

\begin{verbatim}
@misc{corda20,
  author = {Mario Román},
  title = {Corda library, Version 0.1},
  howpublished = {GitHub \url{https://github.com/mroman42/corda}},
  year = {2020}
}

\end{verbatim}

\end{document}
