\documentclass{article}

\usepackage{mintedcode}
\usepackage{chorda}
\usepackage{mydiagrams}

\title{Corda \\ \normalsize{\textsf{version 0.1}}}
\author{Mario Román}

\begin{document}
\maketitle

\textbf{Corda} is a Haskell library that produces string and surface diagrams from descriptions of morphisms in monoidal categories and monoidal bicategories.

\begin{figure}[h]
  \centering
  \assocOne \qquad
  \assocTwo
  \caption{String diagrams generated by Corda.}
\end{figure}

Corda lets you write Haskell code that outputs a TikZ library with your diagrams. This library can be called from your main LaTeX file.

\begin{figure}[h]
  \centering
  \associatorDiagram \quad
  \leftUnitorDiagram \quad
  \rightUnitorDiagram
  \caption{Surface diagrams generated by Corda.}
\end{figure}



\subsection*{Referencing Corda}

As of June 2020, the best way of referencing \textsf{Corda} is by pointing to its GitHub repository.

\begin{verbatim}
@misc{corda20,
  author = {Mario Román},
  title = {Corda library, Version 0.1},
  howpublished = {GitHub \url{https://github.com/mroman42/corda}},
  year = {2020}
}

\end{verbatim}


\subsection*{The Haskell code}

The previous examples have been generated with the following Haskell code.

\begin{verbatim}
-- | This module exemplifies how to use Corda. Some cells are
-- declared and then some diagrams are constructed on top of them.

module Example where

import Corda


-- We will draw a pseudomonoid (C,o,I) in the monoidal bicategory of
-- categories (Cat,x,1), which is the same as a monoidal category. We
-- start by declaring some primitive cells.
c = obj "\\mathbb{C}"
o = morph "\\otimes" [c,c] [c]
i = morph "I" [] [c]
alpha = transf "\\alpha" [[idt c,o],[o]] [[o,idt c],[o]]
alphainv = transf "\\alpha" [[o,idt c],[o]] [[idt c,o],[o]]
lambda = transf "\\lambda" [[i,idt c],[o]] [[idt c],[idt c]]
lambdainv = transf "\\lambda" [[idt c],[idt c]] [[i,idt c],[o]]
rho = transf "\\rho" [[idt c,i],[o]] [[idt c],[idt c]]
rhoinv = transf "\\rho" [[idt c],[idt c]] [[idt c,i],[o]]


main :: IO ()
main = do

  -- We now declare the diagrams. We start by labelling our package;
  -- then, we proceed to list the diagrams that should compose our
  -- library.
  putStrLn
    "\\ProvidesPackage{mydiagrams}[2020/05/09 v0.1 My Diagrams.]"

  putStrLn $ unlines
    [ mkDiagram3D "associatorDiagram" [[[alphainv]],[[alpha]]]
    , mkDiagram3D "leftUnitorDiagram" [[[lambdainv]],[[lambda]]]
    , mkDiagram3D "rightUnitorDiagram" [[[rhoinv]],[[rho]]]
    , mkDiagram2D "assocOne" [[idt c,o],[o]]
    , mkDiagram2D "assocTwo" [[o,idt c],[o]]
    ]
\end{verbatim}

\end{document}
